\documentclass[11pt]{amsart}
\usepackage{geometry}                % See geometry.pdf to learn the layout options. There are lots.
\geometry{letterpaper}                   % ... or a4paper or a5paper or ... 
%\geometry{landscape}                % Activate for for rotated page geometry
%\usepackage[parfill]{parskip}    % Activate to begin paragraphs with an empty line rather than an indent
\usepackage{graphicx}
\usepackage{amssymb}
\usepackage{epstopdf}
\DeclareGraphicsRule{.tif}{png}{.png}{`convert #1 `dirname #1`/`basename #1 .tif`.png}

\title{Category cheat sheet for Haskellers}
\author{Mark Hopkins}
%\date{}                                           % Activate to display a given date or no date
\DeclareMathOperator{\End}{End}
\DeclareMathOperator{\Coend}{Coend}
\DeclareMathOperator{\Hom}{Hom}

\newcommand{\myvec}[1]{\vec{#1}}
\newcommand{\cat}[1]{\mathbf{#1}}
\newcommand{\op}[1]{#1^{\text{op}}}

\newcommand{\blank}{{-}}

%\def\cat{Cat}
\begin{document}
%\newcommand{cat}[1]{\boldmath{#1}}
    \maketitle

    \subsection*{Products}

    $\cat{C}$ has products if
    $
    \cat{C}(C, A \times B) \cong \cat{C}(C, A) \times \cat{C}(C, B)
    $

    \subsection*{Coproducts}

    $\cat{C}$ has coproducts or sums if
    $
    \cat{C}(A + B, C) \cong \cat{C}(A,C) \times \cat{C}(B,C)
    $

    \subsection*{Quotient}
    A quotient of a set is a way of identifying elements.

    \subsection*{Enriched categories}
    This means doing category theory with a category $\mathcal V$ in place of $\cat{Set}$.
    This means there is a $\mathcal V$-object of object of objects and hom-objects instead of hom-sets.

    $\mathcal V$ should be a complete, cocomplete, biclosed monoidal category.
%\section{}
%\subsection{}

    \subsection*{Complete category}
    A category is complete if it has all limits.

    \subsection*{Cocomplete category}
    A category is cocomplete if it has all colimits.

    \subsection*{Profunctor}
    A profunctor on $\cat C$ is a functor $\op{\cat C} \times {\cat c} \to \cat(Set)$

    \subsection*{Ends}
    If $F:\op{\cat C} \times \cat C \to \cat D$
    then we can form the ``diagonal'' product $\prod_{c\in\cat C} F(c,c)$ in $\cat D$.

    Given a morphism $f: c \to c'$ in $\cat C$, there are two ways we can apply it to factors in this product:
    we can apply it covariantly

    \[
        F(c, f): F(c,c) \to F(c', c')
    \]

    or contravariantly

    \[
        F(f, c'): F(c',c') \to F(c', c').
    \]

    $\End F = \int_{\cat C} F = \int_{c\in\cat C} F(c,c)$
    is the subobject for which these give the same answer for all $f$.

    It comes with projection morphisms
    $\int_{c\in\cat C} F(c,c) \to F(c, c)$ for every $c\in \cat{C}$
    satisfying a universal property.

    The universal quantifier lets us approximate the end of a profunctor in Haskell.

    \begin{verbatim}
        data End f = End (forall c. f c c)

        proj :: End f -> f c c
        proj (End fcc) = fcc
    \end{verbatim}

    \subsection*{Coends}
    If $F:\cat C^{\text{op}} \times \cat C \to \cat D$
    then we can form the ``diagonal'' sum $\sum_{c\in\cat C} F(c,c)$ in $\cat D$.

    Given a morphism $f: c \to c'$ in $\cat C$, there are two ways we can apply it to terms in this sum:
    we can apply it covariantly

    \[
        F(1_c, f): F(c,c) \to F(c', c')
    \]

    or contravariantly

    \[
        F(f, 1_{c'}): F(c',c') \to F(c', c').
    \]

    $\Coend F = \int^{\cat C} F = \int^{c\in\cat C} F(c,c)$
    is the quotient of the sum which identifies the two answers for all $f$.

    It comes with injection morphisms
    $F(c, c) \to \int^{c\in\cat C} F(c,c)$ for every $c\in \cat{C}$
    satisfying a universal property.

    The existential quantifier lets us approximate the coend of a profunctor in Haskell.

    \begin{verbatim}
        data Coend f = forall c. Coend (f c c)

        inj :: f c c -> Coend c
        inj fcc = Coend fcc
    \end{verbatim}

    \subsection*{Presheaf}
    A presheaf on $\cat C$ is a functor $\op{\cat C} -> \cat{Set}$.

    \subsection*{Representable functor}
    A covariant representable functor is one isomorphic to $\cat{C}(c, \blank)$.

    In Haskell this means a datatype defined using only products (no sums).
    For instance

    \begin{verbatim}
        data Pair a = Pair a a deriving (Eq, Show, Functor)
    \end{verbatim}

    \verb|Pair a| is isomorphic to \verb|Bool -> a|, so it's representable, and represented by \verb|Bool|.

    You can think of it as an indexable data structure, with $c$ playing the role of the index.
    We also call this the reader monad.

    A contravariant representable functor is one isomorphic to $\cat{C}(\blank, c)$.
    It's an example of a presheaf

    \subsection*{Yoneda lemma}
    The Yoneda lemma says that natural transformations \textit{out of} a representable functor (covariant or
    contravariant)
    are tightly controlled.

    \[
        [\op{\cat{C}}, \cat{Set}]\left(\cat{C}(c, \blank), G\right) \cong G(c)
    \]

    In words, if $F$ and $G$ are presheaves on $\cat C$, and $F$ is represented by $c$, then the natural transformations
    from $F$ to $G$ line up one-to-one with the elements of the set $G(c)$.
    \subsection*{Kan extensions}
    \subsection*{}

\end{document}  